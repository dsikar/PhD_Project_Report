%%%%%%%%%%%%%
%% METHODS %%
%%%%%%%%%%%%%

%This chapter describes in detail the methods for whatever activities were necessary for your project – e.g., data gathering, data analysis, requirements analysis, design, implementation, testing/evaluation, etc. Your choice of methods should be discussed and justified in view of the project objectives, and with reference to the pertinent literature. Report not only what methods you applied in generic terms, but what you actually did: sufficient information about dates and details for your reader to understand how you ran your project, rather than just how one could run any similar project. 

\chapter{Methods}
\label{Methods} 

\section{Simulators}

In this section we examine some simulators currently available to the task of training autonomous systems, to the be able to consider safety aspects of such models.

\subsection{AirSim}

The AirSim \cite{airsim2017fsr} is a simulator for autonomous vehicles, aimed to be a platform for AI research.

The paper was accepted for the Field and Service Robotics 10th International Conference. No other papers seam to deal with simulators, at least in the title. AirSim references 23 articles.

