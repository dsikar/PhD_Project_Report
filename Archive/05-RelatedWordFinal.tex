%%%%%%%%%%%%%%%%%%
%% RELATED WORK %%
%%%%%%%%%%%%%%%%%%

\chapter{Related Work}
\label{related_work}

This chapter reviews the key literature relevant to our research on softmax-based prototype learning and classification uncertainty. We organize the discussion into three main areas: failure detection and misclassification identification, confidence-based prediction methods, and evaluation frameworks for classification reliability.

\section{Failure Detection and Misclassification Identification}

\subsection{Residual-based Error Detection (RED)}

Qiu and Miikkulainen \cite{QiuMiikkulainen2022} proposed Residual-based Error Detection (RED), a framework for detecting misclassification errors in neural networks. RED's key innovation is its ability to provide both a detection score and the uncertainty associated with that score, addressing the unreliability of traditional confidence metrics \cite{Provost1998}.

RED builds upon the RIO (Residual prediction with Input/Output kernel) method \cite{Qiu2020}, extending it to classification through several modifications:
\begin{enumerate}
    \item Extension of RIO's output kernel to handle multi-output vector data (mRIO)
    \item Learning to predict classification correctness using target detection scores $c_i = \delta_{y_i,\hat{y}_i}$
    \item Computing residuals $r_i = c_i - \hat{c}_i$ between target scores and maximum class probabilities
\end{enumerate}

In deployment, RED provides a Gaussian distribution $\hat{c}'_* \sim N(\hat{c}_* + \bar{\hat{r}}_*, \text{var}(\hat{r}_*))$ where the mean serves as the detection score and the variance explicitly represents uncertainty \cite{QiuMiikkulainen2022}.

\subsection{Prototype-based Classification Methods}

\subsubsection{Feature Space Prototyping}

Traditional prototype-based methods operate in the learned feature space. Center Loss \cite{Wen2016} addresses the geometric structure of feature spaces by encouraging intra-class compactness through an auxiliary penalty term:

$$L_C = \frac{1}{2}\sum_{i=1}^{m} \|f(x_i) - c_{y_i}\|_2^2$$

where $f(x_i)$ is the feature vector and $c_{y_i}$ is the learned centroid for class $y_i$. The total loss combines softmax loss with Center Loss: $L = L_{Softmax} + \lambda L_C$.

Prototypical Networks \cite{Snell2017} demonstrate the effectiveness of feature-space prototyping for few-shot learning. They compute class prototypes as arithmetic means of support set embeddings:

$$\mu_k = \frac{1}{|S_k|} \sum_{x_i \in S_k} f_\phi(x_i)$$

Classification is then performed by finding the closest prototype using softmax over negative squared distances.

% \subsubsection{Output Space Prototyping}

% In contrast to feature-space methods, Stadelmann et al. \cite{Stadelmann2021} proposed "Softmax-Prototyping," which operates directly in the network's output space by averaging posterior class-probability vectors. This approach represents a fundamental shift from geometric to probabilistic prototype representation.

% The softmax prototype for class $c$ is computed as:
% $$\mu_c = \frac{1}{N_c} \sum_{i|label(x_i)=c} p_i$$

% where $p_i = \text{softmax}(f(x_i)/T)$ and $T$ is the temperature parameter. This method's effectiveness depends critically on model calibration and the choice of softmax temperature.

\section{Confidence-based Prediction Methods}

\subsection{Learning Model Confidence}

Corbière et al. \cite{Corbiere2019} addressed failure prediction by proposing True Class Probability (TCP) as an alternative to Maximum Class Probability (MCP). TCP is defined as the probability with respect to the true class: $P(Y = y^*|w,x)$. 

Their ConfidNet architecture learns this TCP criterion during training using an $\ell_2$ loss function. ConfidNet builds on features from pre-trained classification models and consistently outperforms traditional confidence measures including MCP, Trust Score \cite{Jiang2018}, and Monte-Carlo Dropout \cite{Gal2016}.

The key insight is that TCP inherently associates lower values with misclassifications and higher values with correct predictions, improving separation between correct and incorrect predictions compared to traditional confidence measures.

\subsection{Uncertainty Quantification}

Monte Carlo Dropout \cite{Gal2016} provides a Bayesian interpretation of dropout, enabling uncertainty estimation by performing multiple forward passes with different dropout masks. This approach treats dropout as a variational approximation to a Gaussian process.

Trust Score \cite{Jiang2018} measures the agreement between a classifier's prediction and a modified nearest-neighbor classifier in the learned feature space, providing an alternative confidence measure that is less susceptible to adversarial examples.

\section{Evaluation Frameworks for Classification Reliability}

\subsection{Unified Evaluation Protocols}

Jaeger et al. \cite{Jaeger2022} identified critical inconsistencies in failure detection evaluation and proposed a unified evaluation protocol. They highlighted three main pitfalls:
\begin{enumerate}
    \item Heterogeneous task definitions
    \item Limited scope of failure sources
    \item Discrepancy between stated purpose and actual evaluation
\end{enumerate}

Their unified framework defines classification output filtering based on Confidence Scoring Functions (CSF) $g(x)$ with threshold $\tau$. The binary failure label is defined as $y_f(x,w,y) = I(y_{cl} \neq \hat{y}_m(x,w))$.

They advocate for Area under the Risk-Coverage-Curve (AURC) as the primary evaluation metric, which intrinsically assesses both classifier accuracy and CSF ranking performance in a single score.

\subsection{Calibration and Temperature Scaling}

The role of model calibration in confidence estimation has been extensively studied \cite{Guo2017}. Modern neural networks are often poorly calibrated, producing overconfident predictions. Temperature scaling provides a simple post-hoc calibration method by optimizing a single parameter $T$ on a validation set to minimize negative log-likelihood.

The relationship between calibration and prototype quality is particularly relevant for softmax-based prototyping methods, where well-calibrated probability distributions are essential for meaningful prototype formation.

\section{Summary and Research Gaps}

The literature reveals a rich landscape of approaches to classification reliability and prototype-based learning. However, several gaps remain:

\begin{itemize}
    \item Limited exploration of the relationship between output-space prototyping and model calibration
    \item Insufficient theoretical analysis of the geometric properties of probability simplex representations
    \item Need for unified evaluation frameworks that consider both feature-space and output-space methods
    \item Limited understanding of how softmax temperature affects prototype quality and downstream task performance
\end{itemize}

Our research addresses these gaps by providing a comprehensive analysis of softmax-based prototype formation and its applications to classification reliability assessment.
